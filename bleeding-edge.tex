\documentclass[a4paper,11pt]{report}
\usepackage[utf8]{inputenc}
\usepackage[T1]{fontenc}
\usepackage[russian]{babel}
\usepackage{wrapfig}
\usepackage{mathtext}
\usepackage{amsmath}
\usepackage{hyperref}
\usepackage{tabularx}
\usepackage{array} 
\usepackage{amsthm}

\title{Исследование причинности случайных событий}
\author{Сергей Кононенко}


\newtheorem{definition}{Определение}
\newtheorem{question}{Вопрос}

\begin{document}

  \maketitle
  
  
  
  \tableofcontents
  
  
  
  \chapter{Постановка задачи}
  
    \section{Преследуемые цели}
    
      В данной работе преследуются следующие цели:
      
      \begin{enumerate}
          \item Разработка математического аппарата для работы со случайными процессами, в которых имеются зависимости между событиями.
      	\item Анализ типов зависимостей между событиями в случайном процессе.
      	\item Получение выражений для оценки параметров случайных событий в процессе.
      	\item Поиск найболее удобного способа представления случайных процессов для обработки и анализа.
      \end{enumerate}
      
      
    
  \chapter{Основные понятия}
  
    \section{Определения}

      \begin{definition}
        \textbf{Параметр объекта} -- некая величина характеризующая свойство объекта.
      \end{definition}
    
      \begin{definition}
        \textbf{Состояние} -- совокупность параметров объекта.
      \end{definition} 
         
      \begin{definition}
        \textbf{Исходное состояние} -- совокупность выбранных в качестве исходных значений параметров объекта.
      \end{definition}

      \begin{definition}
        \textbf{Случайное событие (событие)} -- определенное значение одного или нескольких параметров объекта. Определенное состояние объекта.
      \end{definition}
        
      \begin{definition}
        \textbf{Случайная величина} -- степень различия состояния наблюдаемого элемента с его исходным состоянием.
      \end{definition}

      

  \chapter{Случайный процесс}
  
    Пусть каждому элементу множества значений одного из параметров объекта соответствует совместная плотность распределения остальных параметров (\emph{далее параметров}) объекта. 
    
    Таким образом задан оператор случайного процесса для целевого параметра. 
    
    \begin{definition}
        \textbf{Целевой параметр сулчайного процесса} -- параметр объекта задающий область определения опратора случайного процесса.
      \end{definition}
     
    \section{Оператор случайного процесса}
    
    ...
    
  
    \section{Последовательность событий}
    
      Отображение множества значений целевого параметра случайного процесса на множество натуральных чисел и отображение множества натуральных чисел на множество совместных плотностей распределения параметров объекта обуславливает последовательность случайных событий.
      
      \begin{definition}
        \textbf{Представление случайного процесса} -- композиция отображений множества значений целевого параметра на множество натуральных чисел и отображения множества натуральных чисел на множество плотностей распределения.
      \end{definition}
        
      Отображение множества значений целевого параметра и определяет состояние объекта при случайном событии.
        
        
        
  \chapter{Дальнейшее развитие}
  
    \section{Направления}
      
      \begin{enumerate}
      	\item Вывод определений:
    	    \begin{enumerate}
      	  	\item причины случайного события;
      	  	\item не элементарного события;
          \end{enumerate}
      	\item Параметры случайных величин в случаном процессе;
      	\item Комбинации случайных процессов, не элементарное случайное событие.
      \end{enumerate}
  
\end{document}

    
