\documentclass[a4paper,11pt]{report}
\usepackage[utf8]{inputenc}
\usepackage[T1]{fontenc}
\usepackage[russian]{babel}
\usepackage{wrapfig}
\usepackage{mathtext}
\usepackage{amsmath}
\usepackage{hyperref}
\usepackage{tabularx}
\usepackage{array} 
\usepackage{amsthm}

\title{Исследование причинности случайных событий}
\author{Сергей Кононенко}


\newtheorem{definition}{Определение}
\newtheorem{question}{Вопрос}

\begin{document}

  \maketitle
  
  \tableofcontents
  
  \chapter{Постановка задачи}
  
    \section{Преследуемые цели}
    
      В данной работе преследуются следующие цели:
      
      \begin{enumerate}
      	\item Разработка математического аппарата для работы со случайными процессами, в которых имеются зависимости между событиями.
      	\item Анализ типов зависимостей между событиями в случайном процессе.
      	\item Получение выражений для оценки параметров случайных событий в процессе.
      \end{enumerate}
    
  \chapter{Основные понятия}
  
    \section{Определения}
    
      \begin{definition}
        Объект -- вообще что угодно.
      \end{definition}
    
      \begin{definition}
        Параметр объекта -- некая величина (непрерывная или дискретная) характеризующая свойство объекта.
      \end{definition}
    
      \begin{definition}
        Состояние объекта -- совокупность параметров объекта.
      \end{definition}

      \begin{definition}
        Случайное событие (событие) -- изменение одного или нескольких параметров (состояния) объекта в ограниченный промежуток времени.
      \end{definition}
        
      \begin{definition}
        Случайная величина -- степень изменения состояния наблюдаемого элемента.
      \end{definition}
        
      \begin{definition}
        Случайный процесс -- последовательность случайных величин на выбранном множестве.
      \end{definition}

  \chapter{Дальнейшее развитие}
  
    \section{Направления}
      
      \begin{enumerate}
      	\item Вывод определений:
    	    \begin{enumerate}
      	  	\item причины случайного события;
      	  	\item не элементарного события;
          \end{enumerate}
      	\item Параметры случайных величин в случаном процессе;
      	\item Комбинации событий.
      \end{enumerate}
  
\end{document}

    
